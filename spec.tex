\documentclass{article}

\usepackage{amsmath}
\usepackage[parfill]{parskip}

\title{Detecting similar chunks in an image.}
\author{Tommy Pujol}
\date{\today}


\begin{document}
\maketitle

\section{Introduction}

To find similar chunks we first want to break up the image into chunks.
To do this we first need to know how will a chunk be represented.

To start off, a chunk will consist of an array of $N$ pixels whose value will be
calculated as follows:

\begin{equation}
	p = \sum_{i=0}^{N} x_i
\end{equation}

where $x_i$ is iterating over the values of the tuple with it's corresponding rgb
values. So a chunk $c$ will be something in the form of:

\begin{equation}
	c = \begin{vmatrix}
		p_1 & p_2 & p_3 \\
		p_4 & \dots & p_n
	\end{vmatrix}
\end{equation}

Then we can rewrite the value of a chunk $c$ to be:

\begin{equation}
	c = \frac{\sum\limits_{i=0}^{N} c_i}{N}
\end{equation}

Once we have all the chunks we need we need a good way of determining which chunks
are similar to each other. To do this we can calculate the average difference between
chunks and say that if the difference between two chunks $c_1$ and $c_2$ is less than
half of the mean difference ($c_1 - c_2 \leq \frac{\overline{x}}{2}$) then we can assume
that the chunks are similar.

\begin{equation}
	C = \begin{vmatrix}
		c_1 & c_2 & c_3 \\
		c_4 & \dots & c_n
	\end{vmatrix}
\end{equation}

\begin{equation}
	\overline{x} = \frac{C_N - C_1}{N - 1}
\end{equation}

\footnotetext{For $\overline{x}$ to work the matrix $C$ will need to be sorted from smallest to greatest.}

With this we can now determine which chunks are similar and we can fine-tune the value
to determine which chunks are similar.

\end{document}
